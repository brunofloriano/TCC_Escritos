%TCIDATA{LaTeXparent=0,0,relatorio.tex}

\chapter{Descri��o do conte�do do CD}\label{AnCD}
%O CD que acompanha este trabalho cont�m os c�digos trabalhados na pasta \emph{quad}. O arquivo \textit{main.cpp} � o principal, contendo as fun��es de inicializa��o e a fun��o que define a thread de controle. O arquio \textit{command.cpp} abrange os comandos de escrita e leitura dos motores. Os arquivos \textit{gdatalogger.c}, \textit{gmatlabdatafile.c} e \textit{gqueue.c}


\begin{enumerate}
	\item Programas em C++ para inicializa��es, defini��o da thread de controle, filtragem, medi��o e comandos. A execu��o do arquivo "main" \space executar� estas fun��es na ordem correta.
	\item Programa em C para implementa��o do gDataLogger
	\item Diret�rio com os arquivos do Matlab ".mat" \space obtidos para a aquisi��o de dados
	\item Arquivos do Matlab ".m" \space para a an�lise dos dados. O arquivo "analise\_vmotores\_new.m" \space analisa os dados dos sistemas sem paralelismo (tanto do modo posi��o quando do modo velocidade). O arquivo "analise\_pmotores.m" \space trata os dados do sistema com paralelismo
\end{enumerate}