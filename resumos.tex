%TCIDATA{LaTeXparent=0,0,relatorio.tex}

\resumo{Resumo}{O presente trabalho tem o objetivo principal de desenvolver um sistema de controle de equil�brio para a plataforma quadr�pede do LARA. Neste sentido, deseja-se que o rob� seja capaz de manter a sua estabilidade frente � aplica��o de dist�rbios que, do contr�rio, o fariam cair. Dessa forma, o rob� poderia ser capaz de manter uma marcha constante, desenvolvida em trabalhos anteriores, mesmo com a presen�a de perturba��es. Para implementar o sistema proposto adotou-se, incialmente, um controlador baseado em ganhos de velocidades, isto �, em que cada pata responderia com uma velocidade proporcional ao dist�rbio sofrido. Devido a algumas limita��es dos componentes, modificou-se esta concep��o para o uso de posi��es. Ao final, com o controle implementado, adotou-se um sistema com paralelismo de modo que o rob� voltasse � sua marcha ap�s a corre��o do dist�rbio, em regime permanente.}

\vspace*{2cm}

\resumo{Abstract}{The main purpose of this project is to develop a balance control system for LARA's quadruped platform. In such way, the robot should be capable of mantaining it's stability in response to disturbances that might make it fall. Hence, the robot could be able to keep a regular gait, developed in previous projects, even with the presence of perturbations. To implement the proposed system it was used, at first, a controller based in speed gains, i.e., each leg would respond with an angular speed proportional to the received disturbance. Due to some component limitations, the initial conception was modified for the use of positions. In the end, with the controller implemented, a system with paralelism was used in a way that the robot would return to its gait after the correction of the disturbance, in steady state.}