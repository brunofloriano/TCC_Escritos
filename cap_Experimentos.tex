%TCIDATA{LaTeXparent=0,0,relatorio.tex}
                      
\chapter{Resultados Experimentais}\label{CapExperimentos}

% Resumo opcional. Comentar se n�o usar.
\resumodocapitulo{Resumo opcional.}

\section{Introdu��o}

%Na introdu��o dever� ser feita uma descri��o geral dos experimentos realizados. 

%Para cada experimenta��o apresentada, descrever as condi��es de experimenta��o (e.g., instrumentos, liga��es espec�ficas, configura��es dos programas), os resultados obtidos na forma de tabelas, curvas ou gr�ficos. Por fim, t�o importante quando ter os resultados � a an�lise que se faz deles. Quando os resultados obtidos n�o forem como esperados, procurar justificar e/ou propor altera��o na teoria de forma a justific�-los.
